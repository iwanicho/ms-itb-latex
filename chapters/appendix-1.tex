\chapter{Tabel Peralatan Laboratorium}

\pagestyle{plain}
\begin{center}
\newcolumntype{N}{>{\centering\arraybackslash}p{0.05\textwidth}}
\newcolumntype{P}{>{\raggedright\arraybackslash}p{0.2\textwidth}}
\newcolumntype{K}{>{\raggedright\arraybackslash}p{0.25\textwidth}}

\begin{tabular}{|N|P|c|K|}
\hline
\textbf{No.} & \multicolumn{1}{|c|}{\textbf{Peralatan}} & \textbf{Gambar} & \multicolumn{1}{|c|}{\textbf{Kegunaan}}\\
\hline
1. & Krusibel alumina & \raisebox{-\totalheight}{\includegraphics[width=0.28\textwidth]{resources/cover-ganesha.jpg}} & Digunakan sebagai alas spesimen \textit{green body} dan \textit{pellet} di dalam oven\\
\hline
2. & Kertas perkamen (kertas takar) & \raisebox{-\totalheight}{\includegraphics[width=0.28\textwidth]{resources/cover-ganesha.jpg}}& Digunakan untuk membantu menakar jumlah \textit{powder} LATP yang akan dikompaksi\\
\hline
3. & Spatula & \raisebox{-\totalheight}{\includegraphics[width=0.28\textwidth]{resources/cover-ganesha.jpg}} & Digunakan untuk membantu memindahkan \textit{powder} LATP\\
\hline
4. & Neraca digital & \raisebox{-\totalheight}{\includegraphics[width=0.28\textwidth]{resources/cover-ganesha.jpg}} & Digunakan untuk menimbang massa spesimen\\
\hline
\end{tabular}
\newpage

\begin{tabular}{|N|P|c|K|}
\hline
\textbf{No.} & \multicolumn{1}{|c|}{\textbf{Peralatan}} & \textbf{Gambar} & \multicolumn{1}{|c|}{\textbf{Kegunaan}}\\
\hline
5. & Amplas 2000 & \raisebox{-\totalheight}{\includegraphics[width=0.28\textwidth]{resources/cover-ganesha.jpg}} & Digunakan untuk menghaluskan permukaan spesimen \textit{pellet} LATP\\
\hline
6. & Pipet ukur dengan \textit{filler bulb} & \raisebox{-\totalheight}{\includegraphics[width=0.28\textwidth]{resources/cover-ganesha.jpg}} & Digunakan untuk menakar jumlah larutan PVA yang akan dicampur dengan \textit{powder} LATP\\
\hline
7. & Botol semprot & \raisebox{-\totalheight}{\includegraphics[width=0.28\textwidth]{resources/cover-ganesha.jpg}} & Digunakan unruk menyimpan larutan PVA dan alkohol\\
\hline
\end{tabular}

\end{center}