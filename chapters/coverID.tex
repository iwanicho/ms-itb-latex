\clearpage
\pagestyle{empty}
\begin{center}
%ubah angka {_cm} untuk atur jumlah & pemenggalan baris pada judul
\newcolumntype{T}{>{\centering\arraybackslash}m{13cm}}

\smallskip
\begin{tabular}{T}
    \MakeUppercase{\textbf{\fontsize{16pt}{20pt}\selectfont \titleID}}
\end{tabular}

    \vfill
    \textbf{\fontsize{14pt}{20pt}\selectfont Laporan Tugas Sarjana} \\[1\baselineskip]

    \fontsize{16pt}{20pt}\selectfont
        Oleh \\ \theauthor \, (\nim) \\[2\baselineskip]
        Pembimbing \\ \advisorA \\ \advisorB
    \vfill

    \begin{figure}[h]
        \centering
      	\includegraphics[width=0.2\textwidth]{resources/cover-ganesha.jpg}
    \end{figure}
    \vfill

    \MakeUppercase{
        \fontsize{15pt}{20pt}\selectfont \majorID \\
        \fontsize{18pt}{20pt}\selectfont \facultyID \\
        \fontsize{20pt}{20pt}\selectfont \university \\
    }
    \fontsize{16pt}{20pt}\selectfont 2022

\end{center}

\clearpage
