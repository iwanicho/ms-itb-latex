\clearpage
\addcontentsline{toc}{chapter}{LEMBAR PENGESAHAN}
\pagestyle{plain}

\begin{center}
%ubah angka {_cm} untuk atur jumlah & pemenggalan baris pada judul
\newcolumntype{T}{>{\centering\arraybackslash}m{13cm}}
\smallskip
    \textbf{\fontsize{16pt}{20pt}\selectfont LEMBAR PENGESAHAN \\}
    
    \fontsize{14pt}{20pt}\selectfont Laporan Tugas Sarjana
    \vfill

    \begin{tabular}{T}
        \MakeUppercase{\textbf{\fontsize{16pt}{20pt}\selectfont \titleID}}
    \end{tabular}

    \vfill
    \fontsize{14pt}{15pt}\selectfont
    Oleh \\ \theauthor \\ \nim \\[2\baselineskip]
    \vfill

    \normalsize \normalfont {
    \majorID \\ \facultyID \\ \university \\[2\baselineskip]
    }

    \vfill
    Disetujui pada Tanggal: \thedate

    \vfill
    \setlength{\tabcolsep}{12pt}
    \begin{tabular}{c@{\hskip 0.5in}c}
        Pembimbing 1 & Pembimbing 2 \\
        & \\
        & \\
        & \\
        & \\
        \underline{\advsA} & \underline{\advsB} \\
        NIP \nipA & NIP \nipB \\
    \end{tabular}

\end{center}
\clearpage
