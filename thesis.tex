\documentclass[12pt, a4paper, onecolumn, oneside, final]{report}

%% Pilih Bahasa
\usepackage[main=bahasa,english]{babel}
%-------------------------------------------------------------------%
%
% Konfigurasi dokumen LaTeX untuk laporan tesis IF ITB
%
% @author Petra Novandi
%
%-------------------------------------------------------------------%
%
% Berkas asli berasal dari Steven Lolong
%
%-------------------------------------------------------------------%

%% Ukuran kertas
\special{papersize=210mm,297mm}

%% Setting margin
\usepackage[top=3cm,bottom=3cm,left=3cm,right=2cm,a4paper]{geometry}

% Spacing 1.5
\usepackage{setspace}
\renewcommand{\baselinestretch}{1.5}

% Prevent overfull (or underfull) if possible
\setlength{\emergencystretch}{25pt}

% Avoid widow and orphan lines if possible
\widowpenalty500
\clubpenalty10000

%% Format citation and writings
\usepackage{mathptmx}
\usepackage{newtxtext}		% the Times New Roman font for your document
\usepackage{indentfirst}    % indentasi di awal paragraf
\usepackage{array}
\usepackage[utf8]{inputenc}
\usepackage{csquotes}
\usepackage{graphicx}
\usepackage{titling}
\usepackage{blindtext}
\usepackage{sectsty}
\usepackage{chngcntr}
\usepackage{etoolbox}
\usepackage{hyperref}       	% Package untuk link di daftar isi.
\usepackage{titlesec}       	% Package Format judul
\usepackage[skip=0pt]{parskip}	% untuk menghilangkan indentasi pad abagian tertentu
\usepackage{booktabs}
\usepackage{tabularx}
\usepackage[chapter]{algorithm}
\usepackage{algpseudocode}
\usepackage{comment}
\usepackage{cite}			% kita pakai format dari .bst, bukan biblatex

%% Setting judul
\chapterfont{\centering \fontsize{14pt}{20pt}\selectfont}
\titleformat{\chapter}[display]
    {\fontsize{14pt}{0pt}\selectfont\centering\bfseries}
    {\chaptertitlename\ \arabic{chapter}}{7pt}
    {\fontsize{14pt}{0pt}\selectfont\bfseries}
\titlespacing*{\chapter}{0pt}{1cm}{1\baselineskip} % supaya margin=4cm di halaman judul bab

% Format judul section (dan sub(sub)section)
\titleformat*{\section}{\bfseries\normalsize}
\titleformat*{\subsection}{\bfseries\normalsize}
\titleformat*{\subsubsection}{\bfseries\normalsize}
\titlespacing*{\section}{0pt}{2ex}{0pt}
\titlespacing*{\subsection}{0pt}{2ex}{0pt}
\titlespacing*{\subsubsection}{0pt}{2ex}{0pt}

%% Setting nomor pada subbsubsubbab
\setcounter{secnumdepth}{3}

%%%%%%%%%%%%%%%%%%%%%%%%%%%%%%%%%%%%%%%%%%%%%%%%%%
%  TABLE OF CONTENTS, LISTS OF FIGURES & TABLES  %
%%%%%%%%%%%%%%%%%%%%%%%%%%%%%%%%%%%%%%%%%%%%%%%%%%
\usepackage[titles]{tocloft}
\usepackage[titletoc]{appendix}
\usepackage{tocbibind}

% Judul bab (chapter) di ToC pakai titik-titik halaman
\renewcommand{\cftchapleader}{\cftdotfill{\cftchapdotsep}}
\addtocontents{toc}{\protect\renewcommand{\protect\cftchapdotsep}{\protect\cftdotsep}}

% Kedalaman hierarki maksimum ToC
% (yang masuk ToC hanya sampai subsection: I.1.1.)
\setcounter{tocdepth}{2}

% Hilangkan gap antar-bab di ToC
\setlength{\cftbeforechapskip}{0pt}

% Hilangkan indent sub-bab di ToC
% \setlength{\cftsecindent}{0pt}% Remove indent for \section
% \setlength{\cftsubsecindent}{0pt}% Remove indent for \subsection

% Ubah font ToC jadi normal
\renewcommand{\cftchappagefont}{\normalfont}
\renewcommand{\cftpartfont}{\normalfont}            % \part font in ToC
\renewcommand{\cftchapfont}{\normalfont}            % \chapter font in ToC
\renewcommand{\cftsecfont}{\normalfont}             % \section font in ToC
\renewcommand{\cftsubsecfont}{\normalfont}          % \subsection font in ToC

% Tambah kata "Bab" sebelum nomor bab di daftar isi
% TODO: still problematic when used with list of appendices (uncomment these 4 following lines to reproduce the problem)
\renewcommand{\cftchappresnum}{Bab~} % BAB before number in ToC
\newlength{\mylen} % a scratch length
\settowidth{\mylen}{\bfseries\cftchappresnum\cftchapaftersnum} % extra space
\addtolength{\cftchapnumwidth}{\mylen} % add the extra space

% ubah judul (ToC, LoF, LoT) jadi uppercase
\usepackage{etoolbox}
\patchcmd{\tableofcontents}{\contentsname}{\MakeUppercase\contentsname}{}{}
\patchcmd{\listoffigures}{\listfigurename}{\MakeUppercase\listfigurename}{}{}
\patchcmd{\listoftables}{\listtablename}{\MakeUppercase\listtablename}{}{}

% Hacks to rename all chapter-level titles in ToC
%\renewcommand*\contentsname{DAFTAR ISI}
%\renewcommand*\appendixtocname{DAFTAR LAMPIRAN}
%\renewcommand*\listfigurename{DAFTAR GAMBAR DAN ILUSTRASI}
%\renewcommand*\listtablename{DAFTAR TABEL}

% Pisah daftar lampiran dari ToC
%%%
\renewcommand{\appendixtocname}{DAFTAR LAMPIRAN}

\makeatletter
\let\oldappendix\appendices

\renewcommand{\appendices}{
	\clearpage
	% From now, everything goes to the app file and not to the toc
	\let\tf@toc\tf@app
	\addtocontents{app}{\protect\setcounter{tocdepth}{1}}
	\immediate\write\@auxout{
		\string\let\string\tf@toc\string\tf@app^^J
	}
	\oldappendix
}

\newcommand{\listofappendices}{
	\begingroup
	\renewcommand{\contentsname}{\appendixtocname}
	\let\@oldstarttoc\@starttoc
	\def\@starttoc##1{\@oldstarttoc{app}}
	% Reusing the code for \tableofcontents with different
	%   \contentsname and different file handle app
	\tableofcontents
	\endgroup
	\addtocontents{app}{\protect\renewcommand{\protect\cftchapdotsep}{\protect\cftdotsep}}
}
\makeatother
%%%

% Hilangkan gap antara entri gambar & tabel antarbab di daftar tabel 
% dan daftar gambar (hanya terlihat kalau ada gambar/tabel di >1 bab)
\newcommand*{\noaddvspace}{\renewcommand*{\addvspace}[1]{}}
\addtocontents{lof}{\protect\noaddvspace}
\addtocontents{lot}{\protect\noaddvspace}

%%%%%%%%%%%%%%%%%%%%%%%%%%%%%%%%%%%%%%%%%
%  FLOATS: FIGURES, TABLES, ALGORITHMS  %
%%%%%%%%%%%%%%%%%%%%%%%%%%%%%%%%%%%%%%%%%
% Before:
% ---
% Counter untuk figure dan table.
% \counterwithin{figure}{section}
% \counterwithin{table}{section}
% ---

\usepackage{caption}
\DeclareCaptionFont{fcap}{\fontsize{11}{15}\selectfont}
\DeclareCaptionLabelFormat{scap}{#1 #2\hspace{1.5ex}}
\usepackage[labelformat=scap, font=fcap, labelsep=none, justification=justified, format=hang]{caption}
%\usepackage[labelformat=simple, font=fcap]{subcaption}
%% Hack subfigure cross-ref agar pakai tanda kurung
%%   e.g. Gambar II.2(a), bukan Gambar II.2a
%% (method recommended in subcaption package documentation)
%\renewcommand\thesubfigure{(\alph{subfigure})}

% Counter untuk gambar dan tabel
\renewcommand*{\thefigure}{\arabic{chapter}.\arabic{figure}}
\renewcommand*{\thetable}{\arabic{chapter}.\arabic{table}}

% Jarak spasi antara float dengan teks utama
\captionsetup[figure]{belowskip=-1em}
\captionsetup[subfigure]{belowskip=0pt}
\setlength{\textfloatsep}{2\baselineskip}
\setlength{\intextsep}{2\baselineskip}

% Spasi single di environment table
\AtBeginEnvironment{table}
{\renewcommand{\baselinestretch}{1.0}}

% Font lebih kecil untuk tabel
\AtBeginEnvironment{table}
{\fontsize{10pt}{15pt}\selectfont}

% Spasi single di environment algorithm
\AtBeginEnvironment{algorithm}
{\renewcommand{\baselinestretch}{1.0}}

% Rename "Algorithm" into "Algoritma"
\makeatletter
\renewcommand*{\ALG@name}{Algoritma}
\newcommand{\algorithmname}{\ALG@name}
\makeatother


%%%%%%%%%%%%%%%%%%%%%%%%%
%  MATHS AND EQUATIONS  %
%%%%%%%%%%%%%%%%%%%%%%%%%
\usepackage[fleqn]{amsmath}
\setlength{\mathindent}{20pt}	% equation menjorok seperti paragraf baru
\usepackage{amsfonts}
\usepackage{mathtools}

\newcommand{\starteq}{			% hack untuk menghilankan space sebelum & setelah equation
	\setlength\abovedisplayskip{0pt}
	\setlength\belowdisplayskip{0pt}}

% Counter untuk equation
\renewcommand*{\theequation}{\arabic{chapter}.\arabic{equation}}

% Allow page breaks on long equations
\allowdisplaybreaks[1-4]

% Operator dan notasi custom tambahan
% contoh: argmin dan argmax
\DeclareMathOperator*{\argmax}{argmax}
\DeclareMathOperator*{\argmin}{argmin}
% contoh: notasi bayes p(x | y)
\newcommand{\bayes}[2]{p(#1 \mid #2)\xspace}

%%%%%%%%%%%%%%%%%
%  GANTT CHART  %
%%%%%%%%%%%%%%%%%
\usepackage{changepage}
\usepackage{pgfgantt}
\ganttset{calendar week text={\currentweek}}

%%%%%%%%%%%%%%
%  DIAGRAMS  %
%%%%%%%%%%%%%%
\usepackage{tikz}
\usetikzlibrary{shapes, arrows, fit}
\tikzstyle{block} = [rectangle, draw, minimum height=3em, minimum width=6em, text centered]
\tikzstyle{pinstyle} = [pin edge={to-, thin, black}]
\tikzstyle{input} = [coordinate]
\tikzstyle{output} = [coordinate]

%%%%%%%%%%%%%%%%%%%%%%%%%%%%%%%%
%  GLOSSARY AND ABBREVIATIONS  %
%%%%%%%%%%%%%%%%%%%%%%%%%%%%%%%%
% Load package acronym dan indexonlyfirst untuk hanya 
% menunjukkan kemunculan pertama singkatan
\usepackage[nomain, abbreviations, indexonlyfirst, automake]{glossaries-extra}

% Buat glossary baru khusus untuk lambang
\newglossary[slg]{symbols}{syi}{syg}{Daftar Lambang}

% Buat glossary
\makeglossaries

% Hilangkan judul dari glossary
\renewcommand{\glossarysection}[2][]{}

% Style glossary untuk Daftar Singkatan
\newglossarystyle{daftarsingkatan}{
	% Dasarkan style pada style long3colheader
	\setglossarystyle{long3colheader}
	
	\renewenvironment{theglossary}{\begin{longtable}{p{3cm}p{\glsdescwidth}p{\glspagelistwidth}}}{\end{longtable}}
	
	% Ganti header glossary
	\renewcommand{\glossaryheader}{
		\raggedright{SINGKATAN} &
		\centering{Nama} &
		\raggedright{Pemakaian pertama kali pada halaman}
		\endhead
	}
	
	% Ganti lebar kolom glossary
	\renewcommand{\glsdescwidth}{10cm}
	\renewcommand{\glspagelistwidth}{2.5cm}
	
	% Buat line break dalam sel menjadi 1 spasi
	\renewcommand{\baselinestretch}{1.0} 
	\renewcommand{\arraystretch}{1.5}
	\selectfont
	
	% Ganti isi glossary menjadi singkatan - deskripsi - kemunculan pertama
	\renewcommand{\glossentry}[2]{
		\glsentryitem{##1}\glstarget{##1}{\glossentryname{##1}} 
		& \glossentrydesc{##1}
		& \centering ##2
		\tabularnewline
	}
	
	\renewcommand{\glsgroupskip}{}
}

% Style glossary untuk Daftar Lambang
\newglossarystyle{daftarlambang}{
	% Dasarkan style pada style long3colheader
	\setglossarystyle{long3colheader}
	% Hack untuk mengubah lebar kolom lambang
	\renewenvironment{theglossary}{\begin{longtable}{p{3cm}p{\glsdescwidth}p{\glspagelistwidth}}}{\end{longtable}}
	% Ganti header glossary
	\renewcommand{\glossaryheader}{
		\raggedright{LAMBANG} &
		\centering{Nama} &
		\raggedright{Pemakaian pertama kali pada halaman}
		\endhead
	}
	
	% Ganti lebar kolom glossary
	\renewcommand{\glsdescwidth}{10cm}
	\renewcommand{\glspagelistwidth}{2.5cm}
	
	% Buat line break dalam sel menjadi 1 spasi
	\renewcommand{\baselinestretch}{1.0} 
	\renewcommand{\arraystretch}{1.5}
	\selectfont
	
	% Ganti isi glossary menjadi lambang - deskripsi - kemunculan pertama
	\renewcommand{\glossentry}[2]{
		\glsentryitem{##1}\glstarget{##1}{\glossentryname{##1}} 
		& \glossentrydesc{##1}
		& \centering ##2
		\tabularnewline
	}
	
	\renewcommand{\glsgroupskip}{}
}

\begin{document}
% Hyphenation penalty
\include{hyphenation-id}
\hyphenpenalty=10000
\tolerance=1
\sloppy

%----------------------------------------------------------------%
% DOCUMENT'S INFORMATION
\title{Sintering and Characterization of Solid Electrolyte Li\textsubscript{1.3}Al\textsubscript{0.3}Ti\textsubscript{1.7}(PO\textsubscript{4})\textsubscript{3}}
\newcommand{\titleID}{\textit{Sintering} dan Karakterisasi Elektrolit Padat Li\textsubscript{1.3}Al\textsubscript{0.3}Ti\textsubscript{1.7}(PO\textsubscript{4})\textsubscript{3}}
\date{22 Juni 2022}
\author{Nicholas Putra Rihandoko}
\newcommand{\nim}{13118066}

\newcommand{\advisorA}{Dr. Bentang Arief Budiman, S.T, M.Eng}
\newcommand{\advsA}{Dr. Bentang A. Budiman, S.T, M.Eng}
\newcommand{\nipA}{118110023}

\newcommand{\advisorB}{Dr. Firman Bagja Juangsa, S.T, M.Eng}
\newcommand{\advsB}{Dr. Firman B. Juangsa, S.T, M.Eng}
\newcommand{\nipB}{198601142020121002}

\newcommand{\major}{Mechanical Engineering Study Program}
\newcommand{\majorID}{Program Studi Teknik Mesin}
\newcommand{\faculty}{Faculty of Mechanical and Aerospace Engineering}
\newcommand{\facultyID}{Fakultas Teknik Mesin dan Dirgantara}
\newcommand{\university}{Institut Teknologi Bandung}

%----------------------------------------------------------------%
% FRONTMATTER (chapters sebelum Bab 1)
\pagenumbering{roman}   %halaman pakai nomor romawi
\setcounter{page}{0}
%\clearpage
\pagestyle{empty}
\begin{center}
%ubah angka {_cm} untuk atur jumlah & pemenggalan baris pada judul
\newcolumntype{T}{>{\centering\arraybackslash}m{13cm}}

\smallskip
\begin{tabular}{T}
    \MakeUppercase{\textbf{\fontsize{16pt}{20pt}\selectfont \thetitle}}
\end{tabular}

    \vfill
    \textbf{\fontsize{14pt}{20pt}\selectfont FINAL PROJECT REPORT} \\[1\baselineskip]

    \fontsize{16pt}{20pt}\selectfont
        By \\ \theauthor \, (\nim) \\[2\baselineskip]
        Supervisor \\ \advisorA \\ \advisorB
    \vfill

    \begin{figure}[h]
        \centering
      	\includegraphics[width=0.2\textwidth]{resources/cover-ganesha.jpg}
    \end{figure}
    \vfill

    \MakeUppercase{
        \fontsize{18pt}{20pt}\selectfont \major \\
        \fontsize{16pt}{20pt}\selectfont \faculty \\
        \fontsize{20pt}{20pt}\selectfont \university \\
    }
    \fontsize{16pt}{20pt}\selectfont 2022

\end{center}

\clearpage

\clearpage
\pagestyle{empty}
\begin{center}
%ubah angka {_cm} untuk atur jumlah & pemenggalan baris pada judul
\newcolumntype{T}{>{\centering\arraybackslash}m{13cm}}

\smallskip
\begin{tabular}{T}
    \MakeUppercase{\textbf{\fontsize{16pt}{20pt}\selectfont \titleID}}
\end{tabular}

    \vfill
    \textbf{\fontsize{14pt}{20pt}\selectfont Laporan Tugas Sarjana} \\[1\baselineskip]

    \fontsize{16pt}{20pt}\selectfont
        Oleh \\ \theauthor \, (\nim) \\[2\baselineskip]
        Pembimbing \\ \advisorA \\ \advisorB
    \vfill

    \begin{figure}[h]
        \centering
      	\includegraphics[width=0.2\textwidth]{resources/cover-ganesha.jpg}
    \end{figure}
    \vfill

    \MakeUppercase{
        \fontsize{15pt}{20pt}\selectfont \majorID \\
        \fontsize{18pt}{20pt}\selectfont \facultyID \\
        \fontsize{20pt}{20pt}\selectfont \university \\
    }
    \fontsize{16pt}{20pt}\selectfont 2022

\end{center}

\clearpage

%\clearpage
\addcontentsline{toc}{chapter}{LEMBAR PENGESAHAN}
\pagestyle{empty}

\begin{center}
%ubah angka {_cm} untuk atur jumlah & pemenggalan baris pada judul
\newcolumntype{T}{>{\centering\arraybackslash}m{13cm}}
\smallskip
    \textbf{\fontsize{16pt}{20pt}\selectfont LEMBAR PENGESAHAN \\}
    
    \fontsize{14pt}{20pt}\selectfont Final Project Report
    \vfill

    \begin{tabular}{T}
        \MakeUppercase{\textbf{\fontsize{16pt}{20pt}\selectfont \thetitle}}
    \end{tabular}

    \vfill
    \fontsize{14pt}{15pt}\selectfont
    By \\ \theauthor \\ \nim \\[2\baselineskip]
    \vfill

    \normalsize \normalfont {
    \major \\ \faculty \\ \university \\[2\baselineskip]
    }

    \vfill
   Approved on: \thedate

    \vfill
    \setlength{\tabcolsep}{12pt}
    \begin{tabular}{c@{\hskip 0.5in}c}
        Main Supervisor & Co-Supervisor \\
        & \\
        & \\
        & \\
        & \\
        \underline{\advsA} & \underline{\advsB} \\
        NIP \nipA & NIP \nipB \\
    \end{tabular}

\end{center}
\clearpage

\clearpage
\addcontentsline{toc}{chapter}{LEMBAR PENGESAHAN}
\pagestyle{plain}

\begin{center}
%ubah angka {_cm} untuk atur jumlah & pemenggalan baris pada judul
\newcolumntype{T}{>{\centering\arraybackslash}m{13cm}}
\smallskip
    \textbf{\fontsize{16pt}{20pt}\selectfont LEMBAR PENGESAHAN \\}
    
    \fontsize{14pt}{20pt}\selectfont Laporan Tugas Sarjana
    \vfill

    \begin{tabular}{T}
        \MakeUppercase{\textbf{\fontsize{16pt}{20pt}\selectfont \titleID}}
    \end{tabular}

    \vfill
    \fontsize{14pt}{15pt}\selectfont
    Oleh \\ \theauthor \\ \nim \\[2\baselineskip]
    \vfill

    \normalsize \normalfont {
    \majorID \\ \facultyID \\ \university \\[2\baselineskip]
    }

    \vfill
    Disetujui pada Tanggal: \thedate

    \vfill
    \setlength{\tabcolsep}{12pt}
    \begin{tabular}{c@{\hskip 0.5in}c}
        Pembimbing 1 & Pembimbing 2 \\
        & \\
        & \\
        & \\
        & \\
        \underline{\advsA} & \underline{\advsB} \\
        NIP \nipA & NIP \nipB \\
    \end{tabular}

\end{center}
\clearpage

\input{chapters/statement}
\clearpage
%\chapter*{Abstrak}
\addcontentsline{toc}{chapter}{ABSTRAK}
\pagestyle{empty}
\begin{center}

\newcolumntype{S}{>{\centering\arraybackslash}m{4cm}}
\newcolumntype{M}{>{\centering\arraybackslash}m{7cm}}
\newcolumntype{A}{>{\centering\arraybackslash}m{16cm}}

\begin{tabular}{|S|M|S|}
    \hline
    Judul & \titleID & \theauthor\\ \hline
    Program Studi & Teknik Mesin & \nim \\ \hline    
    \multicolumn{3}{|A|}{\facultyID} \\
    \multicolumn{3}{|A|}{\university} \\ \hline
    \multicolumn{3}{|A|}{Abstrak} \\
    \multicolumn{3}{|p{16cm}|}{\hspace{20pt}
        %taruh abstrak bahasa indonesia di sini
        \blindtext
    } \\ \hline
\end{tabular}
\end{center}
\clearpage
\clearpage
\addcontentsline{toc}{chapter}{ABSTRACT}
\pagestyle{plain}
\begin{center}

\newcolumntype{S}{>{\centering\arraybackslash}m{4cm}}
\newcolumntype{M}{>{\centering\arraybackslash}m{7cm}}
\newcolumntype{A}{>{\centering\arraybackslash}m{16cm}}

\begin{tabular}{|S|M|S|}
    \hline
    Title & \thetitle & \theauthor\\ \hline
    Study Program & Mechanical Engineering & \nim \\ \hline    
    \multicolumn{3}{|A|}{\faculty} \\
    \multicolumn{3}{|A|}{\university} \\ \hline
    \multicolumn{3}{|A|}{Abstract} \\
    \multicolumn{3}{|p{16cm}|}{\hspace{20pt}
        %taruh abstrak bahasa inggris di sini
        \blindtext
    } \\ \hline
\end{tabular}
\end{center}

\clearpage
\input{chapters/forewords}

%----------------------------------------------------------------%
% DAFTAR ISI, GAMBAR, TABEL, LAMBANG, SINGKATAN, LAMPIRAN
\tableofcontents
\renewcommand{\cftchappresnum}{} % remove Bab before index in appendix
\addtolength{\cftchapnumwidth}{-\mylen} % remove the extra space
\listofappendices
{
    \let\oldnumberline\numberline
    \renewcommand{\numberline}{\figurename~\oldnumberline}
    \listoffigures
}
{
    \let\oldnumberline\numberline
    \renewcommand{\numberline}{\tablename~\oldnumberline}
    \listoftables
}
\input{chapters/abbreviations}

%----------------------------------------------------------------%
% MULAI BAB
% Konfigurasi Bab
\pagenumbering{arabic}
\setcounter{page}{1}
\renewcommand{\chaptername}{Bab}
% Untuk menambahkan daftar bab, buat berkas bab misalnya `chapter-6` di direktori `chapters`, dan masukkan ke sini.
\chapter{Pendahuluan}

Bab Pendahuluan secara umum yang dijadikan landasan kerja dan arah kerja penulis tugas akhir, berfungsi mengantar pembaca untuk membaca laporan tugas akhir secara keseluruhan.

\section{Latar Belakang}

Latar Belakang berisi dasar pemikiran, kebutuhan atau alasan yang menjadi ide dari topik tugas akhir. Tujuan utamanya adalah untuk memberikan informasi secukupnya kepada pembaca agar memahami topik yang akan dibahas.  Saat menuliskan bagian ini, posisikan anda sebagai pembaca – apakah anda tertarik untuk terus membaca?

\section{Rumusan Masalah}

Rumusan Masalah berisi masalah utama yang dibahas dalam tugas akhir. Rumusan masalah yang baik memiliki struktur sebagai berikut:

\begin{enumerate}
    \item Penjelasan ringkas tentang kondisi/situasi yang ada sekarang terkait dengan topik utama yang dibahas tugas akhir.
    \item Pokok persoalan dari kondisi/situasi yang ada, dapat dilihat dari kelemahan atau kekurangannya. Bagian ini merupakan inti dari rumusan masalah.
    \item Elaborasi lebih lanjut yang menekankan pentingnya untuk menyelesaikan pokok persoalan tersebut.
    \item Usulan singkat terkait dengan solusi yang ditawarkan untuk menyelesaikan persoalan.
\end{enumerate}

Penting untuk diperhatikan bahwa persoalan yang dideskripsikan pada subbab ini akan dipertanggungjawabkan di bab Evaluasi apakah terselesaikan atau tidak.

\section{Tujuan}

Tuliskan tujuan utama dan/atau tujuan detil yang akan dicapai dalam pelaksanaan tugas akhir. Fokuskan pada hasil akhir yang ingin diperoleh setelah tugas akhir diselesaikan, terkait dengan penyelesaian persoalan pada rumusan masalah. Penting untuk diperhatikan bahwa tujuan yang dideskripsikan pada subbab ini akan dipertanggungjawabkan di akhir pelaksanaan tugas akhir apakah tercapai atau tidak.

\section{Batasan Masalah}

Tuliskan batasan-batasan yang diambil dalam pelaksanaan tugas akhir. Batasan ini dapat dihindari (tidak perlu ada) jika topik/judul tugas akhir dibuat cukup spesifik.

\section{Metodologi}

Tuliskan semua tahapan yang akan dilalui selama pelaksanaan tugas akhir. Tahapan ini spesifik untuk menyelesaikan persoalan tugas akhir. Tahapan studi literatur tidak perlu dituliskan karena ini adalah pekerjaan yang harus Anda lakukan selama proses pelaksanaan tugas akhir. Bila rumusan masalah berbentuk aksional, cantumkan diagram blok dari sistem. Jika tidak, cantumkan diagram alir. Contoh diagram blok dari sistem ditunjukkan pada Gambar \ref{figure:contoh_diagram_blok_sistem}.

\begin{figure}
	\small
	\centering
	\begin{tikzpicture}[auto, node distance=2cm, >=latex']
		% Boks persepsi
		\node (tekssubsistem1) [] {Subsistem 1};
		\node [block, below of=tekssubsistem1, node distance=1cm] (subsubsistem1) {Subsubsistem 1};
		\node [block, below of=subsubsistem1, node distance=1.5cm] (subsubsistem2) {Subsubsistem 2};
		\node[fit=(tekssubsistem1) (subsubsistem2), dashed,draw,inner sep=0.2cm] (subsistem1) {};
		
		% Boks Navigasi
		\node (tekssubsistem2) [right=1.5cm of tekssubsistem1] {Subsistem 2};
		\node [block, below of=tekssubsistem2, node distance=1cm] (subsubsistem3) {Subsubsistem 3};
		\node [block, below of=subsubsistem3, node distance=1.5cm] (subsubsistem4) {Subsubsistem 4};
		\node[fit=(tekssubsistem2) (subsubsistem4), dashed,draw,inner sep=0.2cm] (subsistem2) {};
		
		% Subsistem lain
		\node [block, left of=subsistem1, node distance=5cm] (subsistem3) {Subsistem 3};
		\node [block, above of=tekssubsistem2] (subsistem4) {Subsistem 4};
		
		\draw[->] (subsistem3) -- node {} (subsistem1);
		\draw[->] (subsistem1) -- node {} (subsistem2);
		\draw[<->] (subsistem1) |- node {} (subsistem4);
		\draw[<->] (subsistem4) -- node {} (subsistem2);
	\end{tikzpicture}
	\caption{Contoh Diagram Blok Sistem}
	\label{figure:contoh_diagram_blok_sistem}
\end{figure}

\section{Sistematika Penulisan}

Subbab ini berisi penjelasan ringkas isi per bab. Penjelasan ditulis satu paragraf per bab buku.

\section{Jadwal Pelaksanaan dan Anggaran}

Subbab ini berisi jadwal pelaksanaan tugas akhir dan anggaran pelaksanaan tugas akhir. Contoh jadwal pelaksanaan tugas akhir ditunjukkan pada Gambar \ref{figure:contoh_jadwal_pelaksanaan} dan contoh anggaran pelaksanaan tugas akhir dirangkum dalam Tabel \ref{table:contoh_anggaran}.

\begin{figure}[ht]
	\small
	\centering
	\begin{ganttchart}[
		hgrid,
		vgrid,
		y unit chart=0.5cm,
		y unit title=0.6cm,
		title height=1,
		x unit=1mm,
		time slot format=isodate,
		time slot unit=day]{2020-09-01}{2020-12-31}
		\gantttitlecalendar{year, month, week=1} \\
		\ganttgroup{Grup Aktivitas 1}{2020-09-01}{2020-09-30} \\
		\ganttbar{Aktivitas 1}{2020-09-01}{2020-09-07} \\
		\ganttbar{Aktivitas 2}{2020-09-08}{2020-09-30} \\
		\ganttgroup{Grup Aktivitas 2}{2020-09-21}{2020-12-31} \\
		\ganttbar{Aktivitas 3}{2020-09-21}{2020-10-07} \\
		\ganttbar{Aktivitas 4}{2020-10-15}{2020-11-07} \\
		\ganttbar{Aktivitas 5}{2020-11-15}{2020-12-31} \\
		\ganttgroup{Grup Aktivitas 3}{2020-10-01}{2020-12-31}
	\end{ganttchart}
	\caption{Contoh Jadwal Pelaksanaan Tugas Akhir}
	\label{figure:contoh_jadwal_pelaksanaan}
\end{figure}

\begin{table}[htbp]
	\small
	\centering
	\caption{Anggaran Biaya Pelaksanaan Tugas Akhir}
	\label{table:contoh_anggaran}
	\begin{tabular}{lcrr}
		\toprule
		\multicolumn{1}{l}{\textbf{Hal}} & \multicolumn{1}{l}{\textbf{Satuan}} & \multicolumn{1}{l}{\textbf{Harga Satuan}} & \multicolumn{1}{r}{\textbf{Jumlah}}\\
		\midrule
		\textbf{Grup Keperluan 1} \\
		Keperluan 1 & 1 buah & Rp1.000.000,00 & Rp1.000.000,00 \\
		Keperluan 2 & 1 set & Rp400.000,00 & Rp400.000,00 \\
		\midrule
		\textbf{Grup Keperluan 2} \\
		Keperluan 3 & 1 buah & Rp2.000.000,00 & Rp2.000.000,00 \\
		Keperluan 4 & 2 buah & Rp300.000,00 & Rp600.000,00 \\
		\midrule
		\textbf{Total} & & & Rp4.000.000,00 \\
		\bottomrule
	\end{tabular}
\end{table}

\begin{table}[htbp]
    \small
    \centering
    \caption{Perbandingan Parameter Fabrikasi dan Hasil Konduktivitas Ionik}
    \label{table:anal_kondi}
    \begin{tabular}{lccccc}
    \toprule
    \multicolumn{1}{c}{\multirow{2}{*}{\textbf{Parameter}}} & \multicolumn{5}{c}{\textbf{Spesimen}}\\
    \cmidrule{2-6}
    & \textbf{H1} & \textbf{H2} & \textbf{Ref. A}\cite{wei_fabrication_2017} & \textbf{Ref. B}\cite{wei_fabrication_2017} & \textbf{Ref. C}\cite{wei_fabrication_2017}\\
    \midrule
    Tekanan kompaksi & 20 MPa & 20 MPa & 420 MPa & 420 MPa & 20 MPa\\
    Penggunaan \textit{binder} & PVA & PVA & PVA & - & -\\
    Temperatur \textit{pre-sintering} & 600$^\circ$C & 600$^\circ$C & 650$^\circ$C & 650$^\circ$C & -\\
    Durasi \textit{pre-sintering} & 1 jam & 1 jam & 2 jam & 2 jam & -\\
    Temperatur \textit{sintering} & 1000$^\circ$C & 1000$^\circ$C & 1000$^\circ$C & 1000$^\circ$C & 900$^\circ$C\\
    Durasi \textit{sintering} & 3 jam & 3 jam & 2 jam & 2 jam & 3 jam\\
    $\rho *$ (\%) & 64.72 & 75.55 & 95 & 74 & 90.80\\
    $R_{gr}$ ($\Omega$) & 323.80 & 64.58 & - & - & 28.92\\
    $R_{gb}$ ($\Omega$) & 1707 & 1446 & - & - & 829.60\\
    $R_b$ ($\Omega$) & 2030.80 & 1510.58 & 848 & 7464 & 858.52\\
    $\sigma_{ion}$ (S/cm) & 0.83E-4 & 1.27E-4 & 1.51E-4 & 0.17E-4 & 1.0E-4\\
    \bottomrule
    \end{tabular}
\end{table}

\begin{figure}[ht]
    \centering
    \includegraphics[width=1\textwidth]{resources/cover-ganesha.jpg}
    \longcaption[Kurva hasil uji tekan pada kondisi \textit{elastic-plateau}]{:\\
    \noindent(a) Kurva Tegangan-regangan; (b) Kurva Kekakuan-regangan}
    \label{fig:chart-stiff}
\end{figure}
\input{chapters/chapter-2}
\input{chapters/chapter-3}
\input{chapters/chapter-4}
\input{chapters/chapter-5}

%----------------------------------------------------------------%
% DAFTAR PUSTAKA
\clearpage
\renewcommand*\bibname{DAFTAR PUSTAKA}
\bibliographystyle{config/msIEEEtran}
\bibliography{config/references}

%----------------------------------------------------------------%
% MULAI LAMPIRAN
\clearpage
\pagestyle{empty}
\begin{center}
    \topskip0pt
    \vspace*{\fill}
    \large\textbf{LAMPIRAN}\normalsize
    \vspace*{\fill}
\end{center}
\clearpage

% Format judul bab lampiran 
\titleformat{\chapter}[display]
    {\fontsize{14pt}{0pt}\centering\selectfont\bfseries}
    {\chaptertitlename\ \thechapter}{7pt}
    {\fontsize{14pt}{0pt}\selectfont\bfseries}

\begin{appendices}
    \chapter{Instrumen Pengujian}
\pagestyle{plain}
    \chapter{Rincian Kasus Uji}
\pagestyle{plain}
\end{appendices}

\end{document}